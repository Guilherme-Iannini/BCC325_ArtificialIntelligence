\documentclass{article}
\usepackage[utf8]{inputenc}
\usepackage[margin=1.2in]{geometry}
\usepackage{hyperref}

\usepackage{listings}
\usepackage{xcolor}

\definecolor{codegreen}{rgb}{0,0.6,0}
\definecolor{codegray}{rgb}{0.5,0.5,0.5}
\definecolor{codepurple}{rgb}{0.58,0,0.82}
\definecolor{backcolour}{rgb}{0.95,0.95,0.92}

\lstdefinestyle{mystyle}{
    backgroundcolor=\color{backcolour},   
    commentstyle=\color{codegreen},
    keywordstyle=\color{magenta},
    numberstyle=\tiny\color{codegray},
    stringstyle=\color{codepurple},
    basicstyle=\ttfamily\footnotesize,
    breakatwhitespace=false,         
    breaklines=true,                 
    captionpos=b,                    
    keepspaces=true,                 
    numbers=left,                    
    numbersep=5pt,                  
    showspaces=false,                
    showstringspaces=false,
    showtabs=false,                  
    tabsize=2
}

\lstset{style=mystyle}


\usepackage{tikz}
\usetikzlibrary{positioning}

\usepackage{natbib}
\usepackage{graphicx}
\usepackage{amsmath}

\title{\vspace{-2 cm}Universidade Federal de Ouro Preto \\ BCC 325 - Inteligência Artificial \\ Agentes Lógicos}
\author{Prof. Rodrigo Silva}
\date{}


\begin{document}

\maketitle



\section{Leitura}

\begin{itemize}
    \item Ler o capítulo 5 do Livro\textit{ Artificial Intelligence: Foundations of Computational Agents,  2nd Edition} disponível em \textit{https://artint.info/}
\end{itemize}

\section{Questões teóricas}

\begin{enumerate}
    \item Considere a seguinte base de conhecimento (KB):
    
    \begin{center}
        \begin{align*}
         a & \leftarrow b \wedge c. \\ 
         b & \leftarrow e. \\ 
         b & \leftarrow d. \\ 
         c &. \\ 
         d & \leftarrow h. \\ 
         e &. \\
         g & \leftarrow a \wedge b  \wedge e. \\
         f & \leftarrow h \wedge b. \\  
        \end{align*}
    \end{center}
    

    \begin{enumerate}
        \item Apresente um modelo da base de conhecimento apresentada.
        \item Apresente uma interpretação que não é um modelo da base de conhecimento apresentada.
        \item Mostre como uma prova bottom-up funcionaria para esta base de conhecimento. Apresente todas as consequências lógicas desta KB.
        \item Apresente uma prova top-down para a pergunta $ask$ $g$.
    \end{enumerate}
    
    \item Em IA utilizamos cláusulas definidas para representar algum conhecimento sobre uma determinada aplicação. Por quê é interessante limitar a linguagem de representação à cláusulas definidas?
    
    \item Por quê não é interessante permitir o operador $\vee$ (``ou") quando modelamos um sistema lógico?
    
    \item Se uma proposição $g$ é consequência lógica de uma $KB$, o que podemos dizer sobre $g$?
    
    \item Dada uma base de conhecimento, $KB$, e um conjunto de observações, $\mathcal{O}$, descreva um procedimento de abdução?
    
    \item O que é uma explicação mínima? 
    
    \item Uma derivação utilizando o algoritmo top-down pode entrar em loop infinito? Explique e apresente um exemplo. 
    
\end{enumerate}
   

% O objetivo desta atividade é implementar um agente que, dada uma base de conhecimento, consiga inferir sobre ela. Como de costume, visite o repositório \url{https://github.com/rcpsilva/BCC325_PLE}, que contém o código base para esta atividade e leia atentamente o README.

% No arquivo \textit{knowledge\_base.py} você vai encontrar uma implementação para uma base de conhecimento. Uma base de conhecimento pode conter 3 tipos de declarações (\textit{statements}). Eles são:

% \begin{itemize}
%     \item Clauses: Cláusulas definidas
%     \item Askables: São átomos transientes (cujo valor muda ao longo do tempo) e devem ser consultados periodicamente.
%     \item Assumables: Atómos utilizados no procedimento de abdução.
% \end{itemize}

% Dadas estas informação você deve:
% \begin{enumerate}
%     \item Implementar o método \texttt{botton\_up()} da classe \texttt{LogicalAgent}.
%     \item Implementar o método \texttt{top\_down(query)} da classe \texttt{LogicalAgent}.
%     \item Implementar o método \texttt{explain(atoms)} da classe \texttt{LogicalAgent}.
% \end{enumerate}

% Observações:
% \begin{itemize}
%     \item Todos os métodos devem ser capazes de lidar com \textit{Askables}. 
%     \item Ao tentar provar um \texttt{Askable} o agente deve perguntar ao usuário se ele é verdadeiro.
%     \item Utilize o \textit{script} no arquivo \textit{logical\_agents\_simulation.py} para testar os algoritmos de prova \textit{top\_down} e \textit{botton\_up}
%     \item Utilize o script do arquivo \textit{logical\_abduction\_simulation.py} para testar o método \texttt{explain(atoms)}.
% \end{itemize}


%\bibliographystyle{plain}
%\bibliography{references}
\end{document}

