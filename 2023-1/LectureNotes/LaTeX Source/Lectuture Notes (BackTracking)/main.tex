\documentclass{article}
\usepackage[utf8]{inputenc}
\usepackage[margin=1.2in]{geometry}
\usepackage{hyperref}

\usepackage{listings}
\usepackage{xcolor}

\definecolor{codegreen}{rgb}{0,0.6,0}
\definecolor{codegray}{rgb}{0.5,0.5,0.5}
\definecolor{codepurple}{rgb}{0.58,0,0.82}
\definecolor{backcolour}{rgb}{0.95,0.95,0.92}

\lstdefinestyle{mystyle}{
    backgroundcolor=\color{backcolour},   
    commentstyle=\color{codegreen},
    keywordstyle=\color{magenta},
    numberstyle=\tiny\color{codegray},
    stringstyle=\color{codepurple},
    basicstyle=\ttfamily\footnotesize,
    breakatwhitespace=false,         
    breaklines=true,                 
    captionpos=b,                    
    keepspaces=true,                 
    numbers=left,                    
    numbersep=5pt,                  
    showspaces=false,                
    showstringspaces=false,
    showtabs=false,                  
    tabsize=2
}

\lstset{style=mystyle}


\usepackage{tikz}
\usetikzlibrary{positioning}

\usepackage{natbib}
\usepackage{graphicx}
\usepackage{amsmath}
\usepackage{algorithm}
\usepackage[noend]{algpseudocode}

\title{\vspace{-2 cm}Universidade Federal de Ouro Preto \\ Lecture Notes \\ Backtracking}
\author{Prof. Rodrigo Silva}
%\date{}


\begin{document}

\maketitle


\section{Backtracking}

\begin{algorithm}
    \caption{Backtracking Algorithm}
    \begin{algorithmic}[1]
    \Function{Backtracking}{problem}
        \If{\Call{IsSolution}{problem}}
            \State \textbf{return} problem \Comment{Found a solution}
        \EndIf
        \ForAll{option \textbf{in} \Call{GenerateOptions}{problem}}
            \If{\Call{IsValid}{option}}
                \State \Call{ApplyOption}{option}
                \State result $\gets$ \Call{Backtracking}{problem}
                \If{result $\neq$ \textbf{None}}
                    \State \textbf{return} result \Comment{Found a solution}
                \EndIf
                \State \Call{UndoOption}{option} \Comment{Backtrack}
            \EndIf
        \EndFor
        \State \textbf{return} \textbf{None} \Comment{No solution found}
    \EndFunction
    \end{algorithmic}
    \end{algorithm}


% \section{N-Queen Problem}

% \begin{algorithm}
%     \caption{N-Queens Backtracking Algorithm}
%     \begin{algorithmic}[1]
%     \Function{SolveNQueens}{$n$}
%         \State board $\gets$ empty $n \times n$ chessboard
%         \State queens $\gets$ empty list
        
%         \State \Call{PlaceQueens}{board, queens, 0, $n$}
        
%         \State \textbf{return} queens  \Comment{Found a solution or None if no solution}
%     \EndFunction
%     \\
%     \Function{PlaceQueens}{board, queens, row, $n$}
%         \If{row $\geq$ $n$}
%             \State \textbf{return} \textbf{true}  \Comment{All queens are placed}
%         \EndIf
        
%         \For{$col$ \textbf{in} $0$ to $n-1$}
%             \If{\Call{IsSafe}{board, row, col}}
%                 \State \Call{MarkQueen}{board, row, col}
%                 \State queens.append((row, col))
                
%                 \If{\Call{PlaceQueens}{board, queens, row+1, $n$}}
%                     \State \textbf{return} \textbf{true}  \Comment{Queens placed successfully}
%                 \EndIf
                
%                 \State \Call{UnmarkQueen}{board, row, col}
%                 \State queens.pop()
%             \EndIf
%         \EndFor
        
%         \State \textbf{return} \textbf{false}  \Comment{Cannot place queens in this configuration}
%     \EndFunction
%     \end{algorithmic}
%     \end{algorithm}

\section{Sudoku}

\begin{algorithm}
    \caption{Sudoku Backtracking Algorithm}
    \begin{algorithmic}[1]
    \Function{SolveSudoku}{board}
        \If{\Call{IsBoardComplete}{board}}
            \State \textbf{return} board \Comment{Found a solution}
        \EndIf
        \State row, col $\gets$ \Call{FindEmptyCell}{board}
        \For{$num$ \textbf{in} [1, 2, 3, 4, 5, 6, 7, 8, 9]}
            \If{\Call{IsValidMove}{board, row, col, num}}
                \State board[row][col] $\gets$ num
                \State result $\gets$ \Call{SolveSudoku}{board}
                \If{result $\neq$ \textbf{None}}
                    \State \textbf{return} result \Comment{Found a solution}
                \EndIf
                \State board[row][col] $\gets$ 0 \Comment{Backtrack}
            \EndIf
        \EndFor
        \State \textbf{return} \textbf{None} \Comment{No solution found}
    \EndFunction
    \end{algorithmic}
    \end{algorithm}


\end{document}

