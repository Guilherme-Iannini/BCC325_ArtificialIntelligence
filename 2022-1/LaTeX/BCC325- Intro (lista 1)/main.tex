\documentclass{article}
\usepackage[utf8]{inputenc}
\usepackage[margin=1.2in]{geometry}
\usepackage{hyperref}

\usepackage{tikz}
  \usetikzlibrary{shapes,arrows,fit,calc,positioning}
  \tikzset{box/.style={draw, diamond, thick, text centered, minimum height=0.5cm, minimum width=1cm}}
  \tikzset{line/.style={draw, thick, -latex'}}

\usepackage{listings}
\usepackage{xcolor}

\definecolor{codegreen}{rgb}{0,0.6,0}
\definecolor{codegray}{rgb}{0.5,0.5,0.5}
\definecolor{codepurple}{rgb}{0.58,0,0.82}
\definecolor{backcolour}{rgb}{0.95,0.95,0.92}

\lstdefinestyle{mystyle}{
    backgroundcolor=\color{backcolour},   
    commentstyle=\color{codegreen},
    keywordstyle=\color{magenta},
    numberstyle=\tiny\color{codegray},
    stringstyle=\color{codepurple},
    basicstyle=\ttfamily\footnotesize,
    breakatwhitespace=false,         
    breaklines=true,                 
    captionpos=b,                    
    keepspaces=true,                 
    numbers=left,                    
    numbersep=5pt,                  
    showspaces=false,                
    showstringspaces=false,
    showtabs=false,                  
    tabsize=2
}

\lstset{style=mystyle}


\usepackage{tikz}
\usetikzlibrary{positioning}

\usepackage{natbib}
\usepackage{graphicx}
\usepackage{amsmath}

\title{\vspace{-2 cm}Universidade Federal de Ouro Preto \\ BCC 325 - Inteligência Artificial \\ Introdução}
\author{Prof. Rodrigo Silva}
\date{}

\begin{document}

\maketitle

%\section*{Instruções}

%Cada aluno deve submeter na Plataforma Moodle um arquivo PDF com o nome no formato, \textit{seu\_nome\_intropython.pdf}, contendo:
%\begin{itemize}
%    \item Nome;
%    \item Número de Matrícula;
%    \item Repostas das questões teóricas; e
%    \item Link para o repositório do GitHub que contém o código da atividade prática. 
%\end{itemize}

\section{Leitura}

\begin{itemize}
    \item Introdução à Python - \url{http://antigo.scl.ifsp.edu.br/portal/arquivos/2016.05.04_Apostila\_Python\_-\_PET\_ADS\_S\%C3\%A3o_Carlos.pdf}
    \item List comprehensions - \url{https://pythonacademy.com.br/blog/list-comprehensions-no-python}
    \item Classes em Python - \url{http://pythonclub.com.br/introducao-classes-metodos-python-basico.html}
    \item Capítulos 1 e 2 - Artificial Intelligence: Foundations of Computational Agents,  2nd Edition - \url{http://artint.info/2e/html/ArtInt2e.html}
\end{itemize}


\section{Questões teóricas}

\begin{enumerate}
\item O que é inteligência artificial?

\item Dados dois agentes hipotéticos $A_1$ e $A_2$, naturais ou artificias, defina critérios para determinar qual dos agentes é mais inteligente. Explique também como você avaliaria estes critérios.

\item Uma casa inteligente é uma casa que cuida de si e de seus habitantes. Ela precisa manter as condições ambientais agradáveis aos habitantes e garantir um estoque mínimo de suprimentos. Ela também precisa ser capaz de requisitar reparos a si própria, caso necessário, e controlar o seu próprio gasto tendo em visa os recursos dos habitantes. Defina qual ou quais conhecimentos prévios, habilidades, objetivos/preferências, estímulos (recebidos do ambiente) e experiências este agente deve ter.  

\item No contexto da disciplina de inteligência artificial, defina o que são \textit{agentes}, descreva os seus componentes e suas principais funções.

\item Sobre arquitetura de agentes e controladores hierárquicos, responda:
    \begin{enumerate}
        \item Por quê utilizamos controladores hierárquicos? Explique.
        \item O que queremos dizer quando falamos que camadas mais altas trabalham em uma escala de tempo diferente das camadas mais baixas?
        \item Explique, de forma geral, o que é uma função de transdução e uma função de transdução causal.
        \item Por quê um agente possui um estado de crença?
    \end{enumerate}

\item Quais funções um agente deve implementar?

\item Quais funções um agente hierárquico deve implementar?

%\item Em que contexto utilizamos controladores hierárquicos? Liste vantagens e desvantagens dos controladores hierárquicos.
\end{enumerate}


\section{Atividades Práticas}

\begin{enumerate}
    \item Resolver todos os exercícios da apostila de Python, disponível em: \url{http://antigo.scl.ifsp.edu.br/portal/arquivos/2016.05.04_Apostila\_Python\_-\_PET\_ADS\_S\%C3\%A3o_Carlos.pdf}
    \item Após ler o tutorial em \url{https://pythonacademy.com.br/blog/list-comprehensions-no-python}, considere as seguintes variáveis:
    
    \texttt{nums = [i for i in range(1,1001)]}\\
    \texttt{sentence = "Practice Problems to Drill List Comprehension in Your Head."}
    
    Considerando as variáveis acima, programe a solução para os seguintes problemas:
    
    \begin{enumerate}
    \item Encontre todos os números de 1 a 1000 que são divisíveis por 8
    \item Encontre todos os número de 1 a 1000 que posuem o dígito 6
    \item Conte o número de espaços na string \texttt{sentence}
    \item Remova todas as vogais da string \texttt{sentence}
    \item Encontre todas as palavras da string \texttt{sentence} que tenham menos do que 5 letras.
    \end{enumerate}
    
    % https://towardsdatascience.com/beginner-to-advanced-list-comprehension-practice-problems-a89604851313
    \item Siga o tutorial disponível em \url{http://pythonclub.com.br/introducao-classes-metodos-python-basico.html} para implementar as seguintes classes:
    \begin{enumerate}
        \item \texttt{Pessoa}
        \item \texttt{Calculadora Simples}
        \item \texttt{Calculadora}
        \item \texttt{Pedido}
    \end{enumerate}
    
    \item Implemente um sistema ambiente/agente em que o agente controle o estoque de papel higiênico de um prédio.

\end{enumerate}

%\bibliographystyle{plain}
%\bibliography{references}
\end{document}

